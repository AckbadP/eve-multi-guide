\documentclass{article}
\usepackage{graphicx}
\usepackage{hyperref}
\usepackage{listings}
\usepackage{helvet}
\usepackage{ragged2e}
\usepackage{tikz}
\usepackage{titlesec}
\usepackage{courier}
\usepackage{pdfpages}
\usepackage{xcolor}

%\usepackage{tikz-uml}

\usetikzlibrary{er,positioning, arrows.meta}

\lstset{basicstyle=\footnotesize\ttfamily, language=Java}

\hypersetup{
    colorlinks=true,
    linkcolor=blue,
    filecolor=magenta,      
    urlcolor=blue,
    pdftitle={Overleaf Example},
    pdfpagemode=FullScreen,
    }
\urlstyle{same}

\definecolor{mygray}{rgb}{0.8,0.8,0.8}
\lstset{
  basicstyle=\ttfamily,
  columns=fullflexible,
  breaklines=true,
  %backgroundcolor=\color{mygray},
  }

\graphicspath{ {./images/} }
%%
%% \BibTeX command to typeset BibTeX logo in the docs
%\AtBeginDocument{%
%  \providecommand\BibTeX{{%
%    Bib\TeX}}}

\titleformat{\section}
  {\normalfont\Large\bfseries}
  {\thesection}
  {1em}
  {}
  [{\titlerule[0.8pt]}]

%\titleformat{\subsection}
%  {\normalfont\Large\bfseries}
%  {\thesection}
%  {1em}
  %{}
  %[{\titlerule[0.3pt]}]

\titleformat{\title}
  {\normalfont\Large\bfseries}
  {\thesection}
  {1em}
  {{\titlerule[0.8pt]}}
  [{\titlerule[0.8pt]}]


\title{
  \line(1,0){250} \\
  \textbf{Multiboxing - A Guide} \\
  v1.0.1 \\
  %%\large \textbf{Final Spec. and Postmortem} \\
  \line(1,0){250}
  }
\author{ 
  Ackbad Pappotte \\\
  }


\renewcommand*\contentsname{Table of Contents}

\begin{document}

%%
%% The "title" command has an optional parameter,
%% allowing the author to define a "short title" to be used in page headers.



\maketitle

\section{Introduction}
Multiboxing is the time honored tradition of creating friends to play with in leu of actually befriending people. This guide aims to be a 
relatively comprehensive look at the process, relevant to running both your 2nd character and your 20th character. This means that you 
should not feel obligated to follow everything outlined in this guide. If you're only running 2 charters for example throttling the FPS of
inactive clients isn't particularly helpful for example. If you find anything incomplete or incorrect, please reach out so I may fix the
issue. Finally, this guide is meant to only cover multi boxing, not alts in general. The latter is far outside of the scope of this guide
and as such is left as an exercise for the reader.

\subsection{Quickstart Checklist}
\begin{enumerate}
  \item Create account via a \href{https://www.eveonline.com/signup?invc=79ffb3de-ef43-400b-a568-e45ac72c6715}{refer a friend link}.
  \item Move to Jita, buy \hyperref[generalSkillplan]{this} skill plan and start it.
  \item Add the new character to your multibox profile in the launcher.
  \item Copy over window layout via the \href{https://github.com/kshannoninnes/CopyEveLayoutTool}{CopyEveLayoutTool}.
  \item Install \href{https://github.com/Proopai/eve-o-preview/releases}{EVE-O Preview}.
  \item \href{https://github.com/Proopai/eve-o-preview}{Configure} EVE-O Preview.
\end{enumerate}

\clearpage
\section{Out of Game Setup}

\subsection{The Rules}
What is not allowed:
\begin{itemize}
  \item Input Broadcasting ie. copping one input across multiple clients. Every input can only correspond to one action
        on a single client (the one exception is if you're using \href{https://wiki.eveuniversity.org/Slash_Commands}{slash command} to 
        queue up commands such as when smartbombing shuttles).
  \item No cutting up the client. You can shrink, resize, and reduce the framerate of a client, but you cannot cut it up.
        \begin{center}
          \makebox[\linewidth]{\includegraphics[width=\linewidth]{client_massicure.jpg}}
          An example of a cut up client.
        \end{center}
\end{itemize}
It is worth noting that both of these were once allowed, so particularly old resources and screenshots may reference these now banned techniques.

\subsection{Tools}

\begin{center}
\makebox[\linewidth]{\includegraphics[width=\linewidth]{multiboxing_without_tools.jpeg}}
Multiboxing without a client manager.
\end{center}

\begin{itemize}
  \item \href{https://github.com/Proopai/eve-o-preview}{EVE-O Preview}, Download the latest version \href{https://github.com/Proopai/eve-o-preview/releases}{here} \\
        This is the most popular tool by far because it's both free and build specifically for EVE. It runs on Windows natively, and on Linus
        though Wine. It dose not yet work on Mac.
  \item \href{https://isboxer.com/}{ISBoxer} \\
        This is the main alternative to EVE-O. It is a program that is designed to work across many games, but costs 50\$/year. It
        is generally more powerful letting you configure the resources used by inactive clients, and also allows for bannable techniques
        such as input broadcasting. A guide to setting it up for EVE can be found \href{https://isboxer.com/wiki/EVE:Quick_Start_Guide}{here}.
  \item \href{https://www.guru3d.com/download/rtss-rivatuner-statistics-server-download}{GURU3D RTSS} FPS throttler.\\
        this is a tool to throttle the fps of inactive clients on windows. While windows dose this automatically for minimized clients, if you
        just swap windows it doesn't always happen. This tool also lets you specify the fps of inactive clients. A guide to stetting it up
        for EVE can be found \href{https://www.youtube.com/watch?v=R7YdEPDD_08}{here}.
  \item \href{https://github.com/kshannoninnes/CopyEveLayoutTool}{CopyEveLayoutTool} \\
        \href{https://forums.eveonline.com/t/manually-copy-settings-between-characters-and-accounts/32704}{Instructions to do it manually} \\
        Linux/Steam file location: \lstinline{~/.local/share/Steam/steamapps/compatdata/8500/pfx/drive_c/users/steamuser/AppData/Local/CCP/EVE} \\
\end{itemize}

\subsection{EVE-O Preview Setup}
One of the most important things in multi-boxing is a proper window layout. If you don't want to use eve-o, you can find a good example of how to stack
windows \href{https://www.youtube.com/watch?v=xpiYxq3mpD8}{here}, but this guide assumes you are using eve-o.
\\
Here is a list of settings I recommend enabling, if I don't mention it I recommend having it disabled:
\begin{itemize}
  \item Track client locations - A no-brainer, it ensure a consistent window layout between play secessions.
  \item Minimize inactive EVE clients - It helps reduce the resource requirements of any client you're not actively looking at.
  \item Previews always on top - If you plan to stack windows above your EVE client this is necessary
  \item Show Overlay - Shows client names. Not necessary but nice if you don't have your layout memorized
  \item Highlight active client - Key if you're running many clients, it help you know which one's up at a glance.
\end{itemize}

There are also several setting you should absolutely set that require you to modify the config.json file EVE-O uses. The main two are hotkeys 
for each client and a cycle group. These are one of the largest draws of using EVE-O as it lets you swap clients without the mouse, something 
crucial if you want to multibox in anger. If you open the config.json file (located in the same directory as the EVE-O.exe) you will see examples 
you can copy for your own settings. If you are unsure of the window/character name EVE-O is expecting simply log into the character with EVE-O 
running and it will generate a settings block at the bottom of the json that includes the window name, location, size, etc. you can copy from. 
Personally I like assigning specific clients to keys 1-0 and the cycle button to mouse button 6, but this isn't always ideal. Keep in mind that 
EVE-O will eat any input that swaps clients, so if you assign a client a hotkey you won't be able to for example type with that key while EVE-O is 
running. If you are unsure on how to set any of these settings, the \href{URLhttps://github.com/Proopai/eve-o-preview}{EVE-O github} goes into 
greater detail on how to set them.


\clearpage
\subsection{Hotkeys}

If you want to run more then maybe 2 clients you need to setup and learn to use hotkeys as it allows you 
to minimize mouse movement. Particularly if you're running more then 2-3 clients, often you'll be using your
mouse to switch clients so you'll need to use hotkeys so that you don't need to move your mouse across the
screen and back every time you swap to a client to do some input like activating your guns.
\\
\begin{center}
\makebox[\linewidth]{\includegraphics[width=\linewidth]{example_layout.jpg}}
Example layout
\end{center}
While this is an extreme example, you can see how far the mouse would need to move from selecting a client with your mouse,
click a mod, then select the next client. \\

\noindent My setup: \\

\begin{itemize}
  \item 1-0: EVE-O switch client
  \item Q, W, E, R: High-Slots 1-4
  \item A, S, D: Mid-Slots 1-3
  \item Z, X, C: Low-Slots 1-3
  \item F: Lock Target\
  \item V: D-Scan
  \item T: Launch Drones
  \item G: Drones Engage Target
  \item B: Recall Drones to Drone Bay
  \item Y: Broadcast Shield
  \item H: Broadcast Armor
  \item N: Broadcast Capacitor
  \item Mouse Button 4 (Thumb 1): Overheat Modifier Key (shift)
  \item Mouse Button 5 (Thumb 2): Broadcast Target (ctrl)
\end{itemize}

This setup runs into issues with ships like nestors or curses where you can easily have more then 10
 active mods, it is optimized specifically for running 1-3 ships in a nano-gang. The reach to broadcast
 for reps is not ideal either. You don't need to copy this setup, but you should spend some time to
 think out your hotkeys as the defaults are generally terrible.

\clearpage
\subsection{Window Layout}

Here are two example layouts for eve-o windows. There are many other ways of doing it but these two are the most common and simple.
\\
\begin{center}
  \makebox[\linewidth]{\includegraphics[width=\linewidth]{2_clients_side.png}}
  Clients at the top.
\end{center}
This is the most common setup you'll see in videos and for good reason. It is taking advantage of the usually empty space at the top 
of the game client to position the windows. This space is almost always empty (unless you put the capacitor at the top), and is within 
easy reach of the mouse if you prefer to switch clients that way, being a similar distance from the center of the screen as other EVE 
UI elements. While the clients in this image are relatively large, it's not unreasonable to put 10 or more windows in that area if you 
re-size them properly, as in such cases the windows are less a button to click and more an indicator of which client is active. 

\begin{center}
  \makebox[\linewidth]{\includegraphics[width=\linewidth]{2_clients_top.png}}
  Clients at the top.
\end{center}
This is an alternative setup that I used for many years. Personally I found it better for differentiating at a glance which window is 
which client, but it dose put the windows a bit further out of reach. It dose however ave the advantage of keeping the top left portion 
of the client clear of eve-o character selection window previews, as they will default to that location whenever you open a new client, 
which can block the neocom menu, search bar, and any other elements you keep at the top left corner of the client.
\clearpage
\subsection{Profiles}
Profiles are a tool built into the EVE launcher that let you save and swap client settings/layouts. They also combine with launcher groups
to let you launch multiple clients with a single click, either to the character select or strait to a specific character. To create a profile
go to the EVE launcher, select an account, then the account settings tab.
\\
\begin{center}
  \makebox[\linewidth]{\includegraphics[width=\linewidth]{profile_location.png}}
  Profile Location
\end{center}

Then click on the box to select a profile or create a new one. When you login with this profile you will have the default UI. You can setup
your UI however you want, and then close the client. You can create multiple different profiles this way, each with a distinct layout and 
hotkeys (though personally I recommend keeping those consistent across layouts). You can then assign multiple clients to a launcher group in 
the settings window.
\\
\begin{center}
  \makebox[\linewidth]{\includegraphics*[width=\linewidth]{launcher_group_page.png}}
  The Launcher Groups Page 
\end{center}
Assigning multiple accounts here will let you say launch all your accounts need for a large fleet fight with the fleet fight profile. It is
also very nice when you need to launch all your accounts for a login regard or for extracting (just don't crash your pc trying to launch 50
at once).
\\
\\
Once you have created a profile you like for one account, log into that profile with every other character and account you want to add it to.
Instead of manually setting everything up every time however, you can use the \href{https://github.com/kshannoninnes/CopyEveLayoutTool}{CopyEveLayoutTool}
instead to clone it. Alternatively you can manually copy settings by following the guide \href{https://forums.eveonline.com/t/manually-copy-settings-between-characters-and-accounts/32704}{here}.
There is also \href{https://github.com/FontaineRiant/EANM}{EANM}, a settings manager to clone your settings to different clients more generally.


\clearpage
\subsection{Tips and Tricks}
\begin{itemize}
  \item A personal chat channel for all your alts. Not only dose this make finding character names for trading and sending isk far easier,
        but the MOTD can also be used to save things like a shared overview that is even accessible if you swap computers.
  \item Bright overview colors. If you want to use the mini client windows provided by EVE-O as eyes, it helps to set all pilots to a default
        bright color so they are more visible in the shrunken down window. Personally I use a solid yellow background for all characters that
        have no standing so neutrals are more easily visible but still stand out among reds and blues.
  \item When fleet-warping alts around, you can queue up jumping though the gate mid-warp. Just initiate the fleet warp then for every client
        press jump in the selected item window mid-warp and they will all jump on contact with the gate.
\end{itemize}



\clearpage
\section{Character Setup}
\subsection{Character Creation}
To create an account, you'll first need a recruitment link. You can get the recruitment bonus SP retroactively, but you may as well do it now. 
You can generate your own link \href{https://www.eveonline.com/recruit}{here}, or if you can't be bothered you can use \href{https://www.eveonline.com/signup?invc=79ffb3de-ef43-400b-a568-e45ac72c6715}{mine}.
Aside from the bonus SP, whoever generated the link gets minor bonuses if the character created though it buys PLEX or Omega time with cash.
\\
\\
Next, you want to create a Caldari character with any bloodline and the School of Applied Knowledge background. You do this because it spawns
you closest to Jita. Other races may be slightly better in terms of starting skills, but that's at most 10-15m of extra training so I prefer the
closer Jita start.

\clearpage
\subsection{General Skill Plan} 
For early training, I would prioritize 3 things, flying a stiletto, flying a hound, and core skills. The below skill plan will train those 3 things
reasonably well, but skips drone and gun skills. There are many other more specialized roles you may want to train an alt into as well such as links,
logi, ewar, hauling, etc. But an extra tackle/eyes ship and an extra bomber are the two most universally useful and easily trained into. It may also be 
worth re-mapping to optimize this skill plan, but personally I wouldn't recommend any remap beyond dumping Social for the other 4 attributes for such a 
short plan. If you're going to do so though drop the plan in \href{https://evemondevteam.github.io/evemon/}{EVE Mon} to sort the plan by attribute.
Another thing to watch out for is event drugs, as they are often a cheap way to trade sp for isk. They are the reason Biology IV is one of the first 
you train, and they do make Biology V worth it in the long run. Finally, the best way to buy sp for isk is the training boosters in the NES store. While 
slower, they offer a far better isk/sp ratio then skill injectors (particularly after 5m sp) do, though if you plan to go this rout you certainly want to train Biology V.
\\
\\
\begin{center}  
\hyperref[generalSkillplan]{General Skill Plan}
\end{center}



\clearpage
\subsection{Alpha Training}
If you are planning on training as an alpha, this is the skill plan I would recommend. The main goal is to get you close to flying a hound and stiletto,
with drone and core skills as well as they are so widely applicable. If you are going this rout, do not redeem the 1m sp recruitment bonus to the 
character until after you have finished training, as you can only train to a max of 5m sp on your character sheet, but the bonus sp dose not count
towards this limit until the 1m sp has been redeemed from your redeem queue. Also, if you do not start training immediately you will need to remove
a couple of skills from the end of the queue as it means this plan would put you over 5m sp. The later skills are less important so feel free to drop 
some as necessary. Finally, get a set of +3 implants. They will improve your training time by approximately 30\%, and given that it will take you 5-6 months
to reach 5m sp this is a significant gain for implants that cost ~20m a piece. +4 and +5s are locked behind omega unfortunately. If you get a couple days
of free omega you can inset +5s and retain their benefit while alpha, but free omega is rare enough that this is not worth planning for.
\\
\\
\begin{center}
\hyperref[alphaSkillplan]{Alpha Skill Plan}
\end{center}


\clearpage
\subsection{Skill Farming}
Skill Farming is something you'll need to do if you don't plan on paying for your alt's subscription out of pocket indefinably. While you may want to pay 
for omega initially, eventually you'll run out of skills to train so it becomes worth trading your alt's SP for isk to fund it's subscription. You'll want 
to pick a skill with a long train time, I use Jump Freighters, but Capitol Capacitor Emission Systems is another good option with a shorter training time 
but also fewer requests. You'll want to extract the SP of this skill every month. This can be done remotely so the character dose not need to fly back to 
Jita every month to extract sp. If you want to maximize your skill farming returns you'll need to buy extractors in bulk whenever there is a major sale, 
and by Omega time ideally when there is a big NES sale but if not buy it in 12 month increments. Generally though it is not a good idea to plan on funding
the account purely though skill farming as this also requires you to always sit in a training pod and you need to time the market around unpredictable NES
sales. Instead it is better to treat skill farming as a method of reducing the subscription cost of the account which you can then supplement with other
activities. Ishtar spinning, PI, and mining are all popular options for this, though there are many more avenues you can pursue. PI in particular had just 
been updated with templates, making it easier then ever to set up en mass. You can find some very good user created templates \href{URLhttps://github.com/DalShooth/EVE_PI_Templates}{here}.

\clearpage
\section{Acknowledgments}
Ackbad Pappotte with a shameless self-insert.\\
The fine folks of Noir. for catching my many typos.\\
Dal, for helping me troubleshoot EVE-O/linux on the \href{https://discord.gg/xYt8R9AFXB}{EVE-O discord}. \\
Krombopulous Nathan, for showing up with a lutris install scrip out of nowhere. \\
\\
\\
Want to contribute? Contact Ackbad Pappotte in-game or raise an issue on the \href{https://github.com/AckbadP/eve-multi-guide}{GitHub}.  
\\
%\subsection*{Changelog}

\clearpage
\section*{Appendix A: Links}
\subsection*{Videos}
These are the videos that are linked throughout this guide as well as some that may be helpful but I had no where else to put
\begin{itemize}
  \item \href{https://www.youtube.com/watch?v=dKbQezW0ZwU}{Optomizing for multibox preformance}
  \item \href{https://www.youtube.com/watch?v=Lm4tVwSkBiE}{How to handle information overload}
  \item \href{https://www.youtube.com/watch?v=UpQpgcKSCS4}{EVE-O Preview Video}
  \item \href{https://www.youtube.com/watch?v=xpiYxq3mpD8}{Multiboxing without a client manager}
  \item \href{https://www.youtube.com/watch?v=iC8PwaFf8ck}{Multiboxing 10x marauders}
  \item \href{https://www.youtube.com/watch?v=p5WXd2IkaOc}{Multiboxing 2x nano}
\end{itemize}

\subsection*{Tools}
This is a list of all tools linked throughout this guide.
\begin{itemize}
  \item \href{https://evemondevteam.github.io/evemon/}{EVE Mon}
  \item \href{https://github.com/Proopai/eve-o-preview}{EVE-O Preview}
  \item \href{https://isboxer.com/}{ISBoxer}
  \item \href{https://www.guru3d.com/download/rtss-rivatuner-statistics-server-download}{GURU3D RTSS}
  \item \href{https://github.com/kshannoninnes/CopyEveLayoutTool}{CopyEveLayoutTool}
  \item \href{URLhttps://github.com/FontaineRiant/EANM}{EANM}
\end{itemize}

\subsection*{Other Links}
\begin{itemize}
  \item \href{https://forums.eveonline.com/t/manually-copy-settings-between-characters-and-accounts/32704}{Manually coppying window layout}
  \item \href{https://www.eveonline.com/signup?invc=79ffb3de-ef43-400b-a568-e45ac72c6715}{Refer a friend link}
  \item \href{https://gist.github.com/arillat/2b7e519a69268f519d507d0ed50b9713}{Git repo on running EVE on Arch}
  \item \href{https://github.com/DalShooth/EVE_PI_Templates}{PI templates}
\end{itemize}

\section*{Appendix B: Linux}
\href{https://www.reddit.com/r/Eve/comments/1hqjm4a/linux_lutris_eve_online_eveo_preview/}{Linux Install Guide (Lutris)}\\
As of now there is no known way to run EVE-O and EVE through steam at the same time as proton will only let you run one program 
at a time in a prefix.
\\
\\
Do note some of EVE-O's settings do not work on linux, specifically the one that sets a client's location on startup.

% TODO: transcribe the setup guide here, and get the new script link when it's moved to github
% TODO: include bash script to launch EVE-O preview that can be added to the prefix


%\subsection*{EVE-O Install Guide}
%\href{https://www.reddit.com/r/Eve/comments/1hqjm4a/linux_lutris_eve_online_eveo_preview/}{install guide}
%The above link describes the process, but I'll write out the basic process here as well in case it doesn't work.
%\begin{enumerate}
%  \item Install \href{https://lutris.net/}{Lutris}
%  \item Install EVE via the script \href{https://www.reddit.com/r/Eve/comments/1hqjm4a/linux_lutris_eve_online_eveo_preview/}{linked here} or do the following: \\
%    \begin{enumerate}
%      \item Install the Dec. 2023 release of the \href{https://lutris.net/games/eve-online/}{EVE Launcher}
%      \item 
%    \end{enumerate}
%  \item 
%\end{enumerate}

\section*{Appendix C: Mac}
I don't know much about multiboxing EVE on a mac, so if you have advice please reach out so I can add it. You may want to look at \href{URLhttps://github.com/williamcpierce/Overview}{this tool},
as it is an attempt to create a mac native version of EVE-O, though it is currently still in alpha as of the time of this writing.

\section*{Appendix D: Skill Plans}
Skill plans for alts, let me know if there are any other plans you think should be included.

\subsection*{General Skill Plan}
\phantomsection\label{generalSkillplan}
\begin{lstlisting}
  Science 1
  Science 2
  Science 3
  Cybernetics 1
  Cybernetics 2
  Biology 1
  Biology 2
  Biology 3
  Infomorph Psychology 1
  Infomorph Psychology 2
  Infomorph Psychology 3
  Biology 4
  Spaceship Command 1
  Minmatar Frigate 1
  Minmatar Frigate 2
  Minmatar Frigate 3
  Minmatar Frigate 4
  Minmatar Frigate 5
  Spaceship Command 2
  Spaceship Command 3
  Navigation 1
  Navigation 2
  Evasive Maneuvering 1
  Evasive Maneuvering 2
  Evasive Maneuvering 3
  Evasive Maneuvering 4
  Evasive Maneuvering 5
  Interceptors 1
  Warp Drive Operation 1
  Warp Drive Operation 2
  Warp Drive Operation 3
  Warp Drive Operation 4
  CPU Management 1
  CPU Management 2
  CPU Management 3
  Propulsion Jamming 1
  Propulsion Jamming 2
  Navigation 3
  Afterburner 1
  Afterburner 2
  Afterburner 3
  High Speed Maneuvering 1
  Power Grid Management 1
  Power Grid Management 2
  Shield Upgrades 1
  Mechanics 1
  Hull Upgrades 1
  Hull Upgrades 2
  Hull Upgrades 3
  Hull Upgrades 4
  Spaceship Command 4
  Spaceship Command 5
  Navigation 4
  Interceptors 2
  Interceptors 3
  Interceptors 4
  CPU Management 4
  Power Grid Management 3
  Capacitor Management 1
  Capacitor Management 2
  Capacitor Management 3
  Capacitor Systems Operation 1
  Capacitor Systems Operation 2
  Capacitor Systems Operation 3
  Electronics Upgrades 1
  Electronics Upgrades 2
  Electronics Upgrades 3
  Energy Grid Upgrades 1
  Energy Grid Upgrades 2
  Energy Grid Upgrades 3
  Gunnery 1
  Gunnery 2
  Weapon Upgrades 1
  Weapon Upgrades 2
  Weapon Upgrades 3
  Science 4
  Power Grid Management 4
  Thermodynamics 1
  Thermodynamics 2
  Thermodynamics 3
  High Speed Maneuvering 2
  High Speed Maneuvering 3
  Acceleration Control 1
  Acceleration Control 2
  Acceleration Control 3
  Afterburner 4
  Shield Management 1
  Shield Management 2
  Shield Management 3
  Shield Operation 1
  Shield Operation 2
  Shield Operation 3
  Shield Upgrades 2
  Shield Upgrades 3
  Shield Upgrades 4
  Tactical Shield Manipulation 1
  Tactical Shield Manipulation 2
  Tactical Shield Manipulation 3
  Electronics Upgrades 4
  Electronics Upgrades 5
  Covert Ops 1
  Cloaking 1
  Cloaking 2
  Cloaking 3
  Cloaking 4
  Missile Launcher Operation 1
  Missile Launcher Operation 2
  Light Missiles 1
  Light Missiles 2
  Light Missiles 3
  Missile Launcher Operation 3
  Heavy Missiles 1
  Heavy Missiles 2
  Heavy Missiles 3
  Missile Launcher Operation 4
  Torpedoes 1
  Missile Bombardment 1
  Missile Bombardment 2
  Missile Bombardment 3
  Missile Bombardment 4
  Bomb Deployment 1
  Weapon Upgrades 4
  Missile Launcher Operation 5
  Torpedoes 2
  Torpedoes 3
  Torpedoes 4
  Covert Ops 2
  Covert Ops 3
  Covert Ops 4
  CPU Management 5
  Cynosural Field Theory 1
  Target Navigation Prediction 1
  Target Navigation Prediction 2
  Target Navigation Prediction 3
  Target Navigation Prediction 4
  Warhead Upgrades 1
  Warhead Upgrades 2
  Warhead Upgrades 3
  Warhead Upgrades 4
  Guided Missile Precision 1
  Guided Missile Precision 2
  Guided Missile Precision 3
  Guided Missile Precision 4
  Rapid Launch 1
  Rapid Launch 2
  Rapid Launch 3
  Rapid Launch 4
  Missile Projection 1
  Missile Projection 2
  Missile Projection 3
  Missile Projection 4
  Capacitor Management 4
  Capacitor Systems Operation 4
  Energy Grid Upgrades 4
  Advanced Weapon Upgrades 1
  Advanced Weapon Upgrades 2
  Advanced Weapon Upgrades 3
  Advanced Weapon Upgrades 4
  EM Armor Compensation 1
  EM Armor Compensation 2
  EM Armor Compensation 3
  Explosive Armor Compensation 1
  Explosive Armor Compensation 2
  Explosive Armor Compensation 3
  Thermal Armor Compensation 1
  Thermal Armor Compensation 2
  Thermal Armor Compensation 3
  Kinetic Armor Compensation 1
  Kinetic Armor Compensation 2
  Kinetic Armor Compensation 3
  Mechanics 2
  Mechanics 3
  Mechanics 4
  Mechanics 5
  Hull Upgrades 5
  Armor Layering 1
  Armor Layering 2
  Armor Layering 3
  Armor Layering 4
  Astrometrics 1
  Astrometrics 2
  Astrometrics 3
  Astrometrics 4
  Astrometric Rangefinding 1
  Astrometric Rangefinding 2
  Astrometric Rangefinding 3
  Astrometric Acquisition 1
  Astrometric Acquisition 2
  Astrometric Acquisition 3
  Astrometric Pinpointing 1
  Astrometric Pinpointing 2
  Astrometric Pinpointing 3
  Jury Rigging 1
  Jury Rigging 2
  Jury Rigging 3
  Armor Rigging 1
  Armor Rigging 2
  Armor Rigging 3
  Drones Rigging 1
  Drones Rigging 2
  Drones Rigging 3
  Electronic Superiority Rigging 1
  Electronic Superiority Rigging 2
  Electronic Superiority Rigging 3
  Energy Weapon Rigging 1
  Energy Weapon Rigging 2
  Energy Weapon Rigging 3
  Shield Rigging 1
  Shield Rigging 2
  Shield Rigging 3
  Projectile Weapon Rigging 1
  Projectile Weapon Rigging 2
  Projectile Weapon Rigging 3
  Launcher Rigging 1
  Launcher Rigging 2
  Launcher Rigging 3
  Hybrid Weapon Rigging 1
  Hybrid Weapon Rigging 2
  Hybrid Weapon Rigging 3
  Signature Analysis 1
  Signature Analysis 2
  Signature Analysis 3
  Signature Analysis 4
  Target Management 1
  Target Management 2
  Target Management 3
  Target Management 4
  Long Range Targeting 1
  Long Range Targeting 2
  Long Range Targeting 3
  Long Range Targeting 4
  Gravimetric Sensor Compensation 1
  Gravimetric Sensor Compensation 2
  Gravimetric Sensor Compensation 3
  Ladar Sensor Compensation 1
  Ladar Sensor Compensation 2
  Ladar Sensor Compensation 3
  Radar Sensor Compensation 1
  Radar Sensor Compensation 2
  Radar Sensor Compensation 3
  Magnetometric Sensor Compensation 1
  Magnetometric Sensor Compensation 2
  Magnetometric Sensor Compensation 3
\end{lstlisting}

\clearpage
\subsection*{Alpha Skill Plan}
\phantomsection\label{alphaSkillplan}
\begin{lstlisting}
  Science 1
  Science 2
  Science 3
  Cybernetics 1
  Cybernetics 2
  Biology 1
  Biology 2
  Biology 3
  Spaceship Command 1
  Minmatar Frigate 1
  Minmatar Frigate 2
  Minmatar Frigate 3
  Minmatar Frigate 4
  Power Grid Management 1
  Power Grid Management 2
  Power Grid Management 3
  Shield Management 1
  Shield Management 2
  Shield Management 3
  Shield Management 4
  Tactical Shield Manipulation 1
  Tactical Shield Manipulation 2
  Tactical Shield Manipulation 3
  Tactical Shield Manipulation 4
  Shield Upgrades 1
  Shield Upgrades 2
  Shield Upgrades 3
  Shield Upgrades 4
  CPU Management 1
  Signature Analysis 1
  Signature Analysis 2
  Signature Analysis 3
  Target Management 1
  Target Management 2
  Target Management 3
  Target Management 4
  CPU Management 2
  Long Range Targeting 1
  Long Range Targeting 2
  Long Range Targeting 3
  Astrometrics 1
  Astrometrics 2
  Navigation 1
  Navigation 2
  Navigation 3
  Navigation 4
  Warp Drive Operation 1
  Warp Drive Operation 2
  Warp Drive Operation 3
  Afterburner 1
  Afterburner 2
  Afterburner 3
  High Speed Maneuvering 1
  High Speed Maneuvering 2
  High Speed Maneuvering 3
  Evasive Maneuvering 1
  Evasive Maneuvering 2
  Evasive Maneuvering 3
  Acceleration Control 1
  Acceleration Control 2
  Acceleration Control 3
  Afterburner 4
  Afterburner 5
  Mechanics 1
  Mechanics 2
  Mechanics 3
  Jury Rigging 1
  Jury Rigging 2
  Jury Rigging 3
  Armor Rigging 1
  Armor Rigging 2
  Armor Rigging 3
  Launcher Rigging 1
  Launcher Rigging 2
  Launcher Rigging 3
  Shield Rigging 1
  Shield Rigging 2
  Shield Rigging 3
  Hull Upgrades 1
  Hull Upgrades 2
  Hull Upgrades 3
  Hull Upgrades 4
  Hull Upgrades 5
  Mechanics 4
  Mechanics 5
  EM Armor Compensation 1
  EM Armor Compensation 2
  Explosive Armor Compensation 1
  Explosive Armor Compensation 2
  Kinetic Armor Compensation 1
  Kinetic Armor Compensation 2
  Thermal Armor Compensation 1
  Thermal Armor Compensation 2
  Armor Layering 1
  Repair Systems 1
  Repair Systems 2
  Repair Systems 3
  Shield Operation 1
  Shield Operation 2
  Shield Operation 3
  Electronic Warfare 1
  Electronic Warfare 2
  Electronic Warfare 3
  Electronic Warfare 4
  CPU Management 3
  Propulsion Jamming 1
  Propulsion Jamming 2
  Propulsion Jamming 3
  Propulsion Jamming 4
  Target Painting 1
  Target Painting 2
  Target Painting 3
  Sensor Linking 1
  Sensor Linking 2
  Sensor Linking 3
  Weapon Disruption 1
  Weapon Disruption 2
  Weapon Disruption 3
  Missile Launcher Operation 1
  Missile Launcher Operation 2
  Missile Launcher Operation 3
  Missile Launcher Operation 4
  Light Missiles 1
  Light Missiles 2
  Light Missiles 3
  Heavy Missiles 1
  Heavy Missiles 2
  Heavy Missiles 3
  Torpedoes 1
  Torpedoes 2
  Torpedoes 3
  Torpedoes 4
  Missile Projection 1
  Missile Projection 2
  Rapid Launch 1
  Rapid Launch 2
  Rapid Launch 3
  Rapid Launch 4
  Warhead Upgrades 1
  Warhead Upgrades 2
  Warhead Upgrades 3
  Missile Bombardment 1
  Missile Bombardment 2
  Missile Bombardment 3
  Missile Bombardment 4
  Guided Missile Precision 1
  Guided Missile Precision 2
  Guided Missile Precision 3
  CPU Management 4
  Capacitor Management 1
  Capacitor Management 2
  Capacitor Management 3
  Capacitor Management 4
  Capacitor Management 5
  Capacitor Systems Operation 1
  Capacitor Systems Operation 2
  Capacitor Systems Operation 3
  Electronics Upgrades 1
  Electronics Upgrades 2
  Electronics Upgrades 3
  Electronics Upgrades 4
  Energy Grid Upgrades 1
  Energy Grid Upgrades 2
  Energy Grid Upgrades 3
  Energy Grid Upgrades 4
  Gunnery 1
  Gunnery 2
  Weapon Upgrades 1
  Weapon Upgrades 2
  Weapon Upgrades 3
  Weapon Upgrades 4
  Science 4
  Power Grid Management 4
  Thermodynamics 1
  Thermodynamics 2
  Thermodynamics 3
  Advanced Weapon Upgrades 1
  Advanced Weapon Upgrades 2
  Advanced Weapon Upgrades 3
  Drones 1
  Drones 2
  Drones 3
  Drones 4
  Drones 5
  Light Drone Operation 1
  Light Drone Operation 2
  Light Drone Operation 3
  Light Drone Operation 4
  Light Drone Operation 5
  Amarr Drone Specialization 1
  Caldari Drone Specialization 1
  Gallente Drone Specialization 1
  Gallente Drone Specialization 2
  Caldari Drone Specialization 2
  Amarr Drone Specialization 2
  Minmatar Drone Specialization 1
  Minmatar Drone Specialization 2
  Drone Avionics 1
  Drone Avionics 2
  Drone Avionics 3
  Drone Avionics 4
  Drone Durability 1
  Drone Durability 2
  Drone Durability 3
  Drone Interfacing 1
  Drone Interfacing 2
  Drone Interfacing 3
  Drone Navigation 1
  Drone Navigation 2
  Drone Navigation 3
  Drone Navigation 4
  Drone Sharpshooting 1
  Drone Sharpshooting 2
  Drone Sharpshooting 3
  Drone Sharpshooting 4
  Medium Drone Operation 1
  Medium Drone Operation 2
  Medium Drone Operation 3
  Heavy Drone Operation 1
  Heavy Drone Operation 2
  Heavy Drone Operation 3
  Gallente Frigate 1
  Gallente Frigate 2
  Gallente Frigate 3
  Gallente Destroyer 1
  Gallente Destroyer 2
  Gallente Destroyer 3
\end{lstlisting}

%\subsection*{Links}
%\subsection*{Cyno}
%\subsection*{Hauling}
%\subsection*{Logi}
%\subsection*{Dreadnought}


\clearpage
\begin{center}
  \vspace*{\fill}
  %{\Huge \ttfamily O7}\\
  \textsl{\large \ttfamily Fly Dangerous O7}
  \vspace*{\fill}
\end{center}

\end{document}
\endinput